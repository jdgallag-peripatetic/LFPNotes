\documentclass{article}



\title{Locally finitely presentable categories notes}
\author{Jonathan Gallagher}


%%%%%%%%%%%%%%%%%%%%%%%%%%%%%%%%%%%%%%%%%%%%
% Package imports
%%%%%%%%%%%%%%%%%%%%%%%%%%%%%%%%%%%%%%%%%%%%

% Etex purportedly makes tex more stable
\usepackage{etex}

% A text colorization package
\usepackage[dvipsnames]{xcolor}
% Growing to and froms
\usepackage{BarrTo}
% Use the whole page
% Make links jump-able
\usepackage[colorlinks,allcolors=MidnightBlue]{hyperref} 
% Standard math packages
\usepackage{amsfonts,amsmath,amssymb,amsthm}
% For proof trees
\usepackage{proof}
% For commutative diagrams
\usepackage{tikz-cd}

% General tikz setup
\usepackage{tikz}
% For zig-zags in diagrams
\usetikzlibrary{decorations.pathmorphing}
% For squiggly lines!!!
\tikzset{snake it/.style={decorate, decoration=snake}}
% For cool arrow heads
\usetikzlibrary{arrows,calc}  

% Because sometimes tikz is annoying
\usepackage[all]{xy}

% Apparently, microtype makes everything look better
\usepackage{microtype}


%%%%%%%%%%%%%%%%%%%%%%%%%%%%%%%%%%%%%%%%%%%%
% Macros
%%%%%%%%%%%%%%%%%%%%%%%%%%%%%%%%%%%%%%%%%%%%
% Nullary commands

\newcommand{\R}{\mathcal{R}}
\newcommand{\tpsh}{\mathsf{TPsh}}
\newcommand{\ev}{\mathsf{ev}}
\newcommand{\x}{\times}
\newcommand{\ww}{\mathcal{W}_1}
\newcommand{\ox}{\otimes}
\newcommand{\blank}{\underline{~}}
\newcommand{\N}{\mathcal{N}}
\newcommand{\X}{\mathbb{X}}
\newcommand{\set}{\mathsf{Set}}
\newcommand{\cvs}{\mathsf{CVS}}
\newcommand{\sman}{\mathsf{SMan}}
\newcommand{\freS}{\textsf{Fr\'{e}-S}}
\newcommand{\freM}{\textsf{Fr\'{e}Man}}
\newcommand{\weil}{\mathsf{Weil}}
\newcommand{\frol}{\textsf{Fr\"{o}l}~}
\newcommand{\weilS}{\mathsf{Weil}\text{-}\mathsf{S}}
\newcommand{\proves}{\vdash}
\newcommand{\<}{\left\langle}
\renewcommand{\>}{\right\rangle}
\newcommand{\id}{\mathsf{Id}}
\newcommand{\Y}{\mathbb{Y}}
\newcommand{\act}{\otimes_\infty}
\newcommand{\microl}{\mathsf{Microl}}
\newcommand{\microlexp}{\mathsf{Microl\text{-}Exp}}
\newcommand{\Mod}{\mathsf{Mod}}
\newcommand{\smicrol}{\mathsf{SMicrol}}
\renewcommand{\lim}{\mathsf{lim}}
\newcommand{\colim}{\mathsf{colim}}
\newcommand{\yoneda}{\mathcal{Y}}
\newcommand{\cur}{\mathsf{cur}}
\newcommand{\frolicher}{Fr\"{o}licher~}   
\newcommand{\ex}{\mathsf{ex}}
\newcommand{\inv}{\text{-}}
\newcommand{\pull}{\mathsf{Pull}}
\newcommand{\Q}{\mathscr{Q}}
\newcommand{\bun}{\mathsf{bun}}
\newcommand{\bunD}{\mathsf{bun}_{\mathcal{D}}}
\newcommand{\dbun}{\mathsf{DBun}}
\newcommand{\V}{\mathcal{V}}
\newcommand{\C}{\underline{C}}
\newcommand{\splitd}{\mathsf{Split}_{\mathcal{E}_{\mathcal{D}}}(\X)}
\newcommand{\fun}{\mathsf{Fun}}
\newcommand{\tang}{\mathsf{Tan}}
\newcommand{\OR}{\mathcal{O}(\R)}

% to make a small vertical bar
% \newcommand{\vertrule}[1][1ex]{\rule{.4pt}{#1}}
% \newcommand{\module}{{\rightarrow \!\!\!\!\!  \vertrule{}}}

\newcommand{\minus}{\scalebox{0.5}[1.0]{\( - \)}}
\newcommand{\betaT}{\underline{\beta}}
\newcommand{\betaTA}{\underline{\beta_A}}
\newcommand{\betaTAN}{\underline{\beta\eta_A}}
\newcommand{\etaT}{\underline{\eta}}
\newcommand{\betaetaT}{\underline{\beta\eta}}
\newcommand{\fv}{\mathsf{fv}}
\renewcommand{\split}{\mathsf{Split}}
\newcommand{\cak}{\mathsf{k}}
\newcommand{\cas}{\mathsf{s}}
\renewcommand{\d}{\mathsf{d}}
\newcommand{\LambdaA}{\Lambda_A}
\newcommand{\LambdaD}{\Lambda_{\beta \partial}}
\newcommand{\M}{\mathcal{M}}
\newcommand{\CC}{\mathcal{C}}
\newcommand{\T}{\mathcal{T}}
\newcommand{\mws}{\mathsf{Mod}(\mathcal{W}_1,\set)}
\newcommand{\sw}{\mathsf{sw}}
\newcommand{\Th}{\mathsf{Th}}
\newcommand{\bind}{\! \leftslice \!}
\newcommand{\symm}{\smile}
\newcommand{\devs}{\!\!\! \begin{tikzcd}[ampersand replacement=\&] {} \arrow{r}[anchor=center]{\bigcirc} \& {} \end{tikzcd}\!\!\!}
\newcommand{\crdevs}{\!\!\!\begin{tikzcd}[ampersand replacement=\&] {} \arrow{r}[anchor=center]{\cap} \& {} \end{tikzcd}\!\!\!}
\newcommand{\sdevs}{\!\!\!\begin{tikzcd}[ampersand replacement=\&] {} \arrow{r}[anchor=center]{\cup} \& {} \end{tikzcd} \!\!\!}
\newcommand{\modto}{{\to}_{/\sim}}
\newcommand{\nf}{\mathsf{nf}}


% Unary commands
\newcommand{\w}[1]{\ensuremath{{#1}\text{-}\mathcal{W}}_1}
\newcommand{\W}[1]{
    \ensuremath{
        {#1}\text{-}\mathsf{Weil}
    }
 }
\newcommand{\den}[1]{{\left\llbracket {#1} \right\rrbracket}}
\newcommand{\norm}[1]{\left| \left| {#1} \right| \right|}

% 
% \newcommand{\rightarrow}{\rightarrow\mathrel{\mkern-14mu}\rightarrow}

\newcommand{\xtwoheadrightarrow}[2][]{%
  \xrightarrow[#1]{#2}\mathrel{\mkern-20mu}\rightarrow
}


% Binary commands 
\renewcommand{\deduce}[2]{\infer{#2}{#1}}
\newcommand{\stacktensor}[2]{\overset{#1}{\underset{#2}{\ox}}}

% 3-ary
\newcommand{\subst}[3]{{#3}[#1/#2]}
\newcommand{\diffl}[3]{ \frac{\partial {#1}}{\partial {#2}}\cdot {#3}}
\newcommand{\diffp}[3]{ \frac{\partial {#1}}{\partial #2}\cdot {#3}     }

% 4-ary commands
\newcommand{\diff}[4]{\frac{\partial {#1}}{\partial {#2}}({#3})\cdot {#4} }
\newcommand{\dtower}[4]{\frac{#1}{#2}(#3)\cdot (#4)}


%%%%%%%%%%%%%%%%%%%%%%%%%%%%%%%%%%%%%%%%%%%%
% Theorems
%%%%%%%%%%%%%%%%%%%%%%%%%%%%%%%%%%%%%%%%%%%%

\newtheorem{theorem}{Theorem}[section]
\newtheorem{obs}[theorem]{Observation}
\newtheorem{proposition}[theorem]{Proposition}
\newtheorem{corollary}[theorem]{Corollary}
\newtheorem{rmk}[theorem]{Remark}
\newtheorem{eg}[theorem]{Example}
\newtheorem{lemma}[theorem]{Lemma}
\newtheorem{conjecture}[theorem]{Conjecture}
\newtheorem{defn}[theorem]{Definition}

\begin{document}
\maketitle

This document does not serve as anything particularly new.  This is for my benefit,
but these notes have been influenced by and are indeed a summary of content in 

\textcolor{violet}{\bf add in bib data and cite the following}
\begin{itemize}
    \item Adamek and Rosick\'{y} 
    \item Coequalizers and free triples
    \item  Dubuc free monoids
    \item Categories of continuous functors
    \item A unified treatment of transfinite constructions for free algebras, free monoids, colimits, associated sheaves, and so on.
    \item Ehresmann and Bastiani
    \item Gabriel-Ulmer
    \item Cosmoi of internal categories
    \item Reflective subcategories, localizations and factorization systems
    \item Localisation of locally presentable categories
    \item Orthogonal subcategories and localization ala nlab page actual cat of fractions
    \item Structures defined by finite limits in the enriched context
    \item A reflection theorem for categories
    \item Algebra valued functors in general and tensor products in particularly (Freyd)
    \item Closed categories generated by commutative monoids (Kock)
    \item The Day convolution paper (find reference you lazy bones)
    \item Barr and Well Category theory for computing science
    \item Monads and Lawvere theories (Hyland) -- (ordinary algebraic theories)
    \item Models of an algebraic theory are regular (need reference)
    \item When is a variety a topos (Johnstone)
    \item Some characterizations of various classes of locally presentable categories (Carboni, Pedicchio, Rosick\'{y})
    \item Reference needed for horn theory/partial equation theory, and essentially algebraic theory, etc ... I guess this is all in A\&R.
\end{itemize}

\section{Idea}

Here are some ideas to keep in mind:
\begin{itemize}
    \item Every set is the union of its finite subsets (indeed its singleton subsets);
    \item Every vector space is the span of its finite dimensional subspaces (indeed every vector has a basis of vectors);
    \item Every presheaf is the filtered colimit of a a finite colimit of representables.
\end{itemize}
We will return to the last example soon.  We will also introduce a few additional 
examples as we go that are of particular interest to me.

These examples can also be rephrased 
in terms of models of a theory.  In general locally finitely presentable 
categories are within equivalence categories of models of a finite limit 
theory.  We will explore this in more detail in the sections that come.

In producing these notes we will stick to categorical considerations for as 
far as possible, and then consider theories of different kinds later.  These 
notes are meant to be comprehensive and will cover background material but 
does assume category theory as per MacLane or Herrlich and Strecker.

\section{Adjunctions and reflective subcategories}

\section{The Yoneda Lemma and Density Theorem}
Although the Yoneda lemma and density theorem are certainly covered in 
MacLane they play such an important role that we thought we might cover them.
We will also introduce the category of elements, and a few additional properties 
about presheaves.

\section{Filtered colimits and limit commutation}
The main goal of this section is to prove that a colimit will commute 
with all finite limits if and only if it is a filtered colimit.  As a 
corollary, any finite limit preserving functor into set, viewed as a colimit 
of representables must be a filtered colimit of those representables.



\section{LFP cats and generators}

\section{LFP cats, orthogonality, and the small object argument}

\section{Models of a small complete category}

\section{Models of a sketch or finite limit theory}

\section{Finite colimit completions}

\section{Localization}

\section{Characterization theorem of LFP cats}

\section{Algebraic theories}
Finitary monads, regularity, when a topos

\section{Commutative theories and Day convolution}

\section{Closed categories of models}

\section{Properties}
Slices

\section{Various classes}
epi-reflective
regular
exact
extensive 



\end{document}